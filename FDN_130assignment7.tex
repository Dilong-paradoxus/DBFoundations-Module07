\documentclass{article}
\usepackage[T1]{fontenc}
\usepackage[lining,light]{InriaSans}
\renewcommand*\oldstylenums[1]{{\fontfamily{InriaSans-OsF}\selectfont #1}}
\let\oldnormalfont\normalfont
\def\normalfont{\oldnormalfont\mdseries}
\usepackage[margin=1.5in,includehead,includefoot,]{geometry}
\usepackage{hyperref}
\usepackage[natbib=true, style=numeric,sorting=none]{biblatex}
\addbibresource{bibliography.bib}


\title{Assignment 07 - Functions \\
\large{IT FDN 130A}}

\author{Nicholas Thibault}
\date{2025-06-04}
\Large{}

\begin{document}

\maketitle

\section*{Introduction}
In this assignment I worked with functions which can reduce repetition of common tasks.

\href{https://github.com/Dilong-paradoxus/DBFoundations-Module07}{Github Link}
\section*{When to Use a SQL User-Defined Function}
There are many use cases for user-defined functions because there are many, many operations someone may want to repeat that aren't and realistically couldn't be included in the database engine. A simple example is applying operations to multiple columns (such as a math equation with several steps) and outputting as another column. A more complex example would be emulating the functionality of check constraints across multiple tables instead of between rows in a single table. 
\section*{Kinds of Functions}
Three categories of functions are Scalar, Inline, and Multi-Statement functions. Scalar functions return a single result value.\cite{SQLscalarG4g} Inline and multi-statement functions return that result in the form of a table which can be used like any other table.\cite{SQLinlineg4g} Multi-statement functions can perform more complex operations, while inline functions must be composed of a single select statement.\cite{SQLmultistatementg4g}
\section*{Summary}
In summary, there are several types of functions which have varying levels of complexity for their various use-cases.
\printbibliography
\end{document}
